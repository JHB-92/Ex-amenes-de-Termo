\documentclass[12pt]{article}
\usepackage[utf8]{inputenc}
\usepackage[spanish]{babel}

%Fuente (compilarlo en Latex.pdf normal porque va más rápido y luego
%y al final insertarlo en Arial) DA ERROR
%\usepackage{fontspec}
%\setmainfont{Arial}

%Estructura de la Página
\usepackage[left=2.54cm,right=2.54cm,top=2.54cm,bottom=2.54cm]{geometry}

%Páginas en horizontal


%Pies de página y encabezado
\usepackage{fancyhdr}

%Comentario párrafos
\usepackage{verbatim}

%Paquete arquitectura página
\pagestyle{fancy}
\fancyhf{}
\lhead{José Honrubia Blanco}
\rhead{Exámenes Ingeniería Térmica.}
\rfoot{\thepage}
\lfoot{Academia David Martínez}

%Interlineado
\renewcommand{\baselinestretch}{0.5}



%Paquete Matemático
\usepackage{amsmath}
\usepackage{amsfonts}
\usepackage{amssymb}
\usepackage{breqn}

%Paquete para el código
% Paquetes Listing

\usepackage{listings}
\usepackage{xcolor}

%Settings 

\definecolor{codegreen}{rgb}{0,0.6,0}
\definecolor{codegray}{rgb}{0.5,0.5,0.5}
\definecolor{codepurple}{rgb}{0.58,0,0.82}
\definecolor{backcolour}{rgb}{0.95,0.95,0.92}


\lstdefinestyle{mystyle}{
 backgroundcolor=\color{backcolour},   
 commentstyle=\color{codegreen},
 keywordstyle=\color{magenta},
 numberstyle=\tiny\color{codegray},
 stringstyle=\color{codepurple},
 basicstyle=\ttfamily\footnotesize,
 breakatwhitespace=false,         
 breaklines=true,                 
 captionpos=b,                    
 keepspaces=true,                 
 numbers=left,                    
 numbersep=5pt,                  
 showspaces=false,                
 showstringspaces=false,
 showtabs=false,                  
 tabsize=2
}

\lstset{style=mystyle}
%Paquete Imágenes

\usepackage{graphicx}
\graphicspath{ {img/} }
\usepackage{subfig}

%Bibliografía
\usepackage[backend=bibtex]{biblatex}
\addbibresource{referencias.bib}

%Espacio entre párrafos
\setlength{\parskip}{0.5cm}
%Sangría
\setlength{\parindent}{0cm}

\title{Exámenes Ingeniería Térmica}
\author{José Honrubia Blanco }
\date{Noviembre 2022}

\begin{document}

\section{Examen Julio 2014}
\subsection{TEORÍA (NO HACE FALTA USAR LAS TABLAS)}
\textbf{Pregunta 1}. (1 punto) Un dispositivo pistón-cilindro contiene 50kg de agua a 250 kPa y 25 ºC. El área de la seeción transversal del pistón es 0,1 m^{2}. Transferimos calor al agua, haciendo que parte de ella se evapore. Cuando el volumen alcanza los 0.2 m^{3}, el pistón choca con un muelle lineal cuya constante elástica es 100 kN/m. Seguimos transfiriendo calro al agua hasta que el pistón se eleva 20 cm más. Determine: 1. Representar el proceso en un diagrama P-V, 2. presión y temperatura final y 3. trabajo realizado durante este proceso. 
\\ 
\textbf{Pregunta 2}. (1 punto) Un motor de un buque funciona según un ciclod e Carnot que extrae calor del agua del mar a 18 ºC y cede una parte a un depósito de hielo seco a $-$78 ºC. Sie l motor debe desarrollar 8000 CV de potencia, ¿cuánto hielo seco se consumirá durante la marcha de un día? El calor de sublimación del hielo seco es de 137 cal/g.
\\
\textbf{Pregunta 3}. (1 punto) sor toma 1 kg/s de aire a 1 bar a 25 ºC comprimiéndolo hasta 8 bar y 160 ºC. La transferencia de calor a su entorno es de 100 kW. Calcular la potencia consumida (kW) y defínase y evalúese la eficiencia exergética.
\\
\textbf{Pregunta 4}. (1 punto) Se introduce helio en una tobera adiabática que oepra en estado estacionario. Las condiciones a la entrada de la tobera son: 1300 K, 4 bar y velocidad de 10 m/s. A la salida, la temperatura del helio es de 900 K, la presión de 1.45 bar. Determinar: 
1. la velocidad a la salida (m/s), 2. rendimiento de la tobera, 3. exergía destruida al pasar el gas a través de la tobera (kJ/kg). \\
\textbf{Hipótesis:} Considerar el helio gas perfecto, $T_{0} = 20ºC$ y $P_{0} = 1 bar$.

\subsection{PROBLEMAS}
\textbf{Problema 1}. (\textbf{2 puntos}) Un cilindro rígido y adiabático, dispuesto horizontalmente, está dividido en dos compartimentos por un émbolo muy delgado y aislante, sujeto inicialmente por unos topes. El compartimento A (el de la izquierda)tiene 10 cm de longitud
y contiene iniucialmente agua a 20 bar y 250 ºC. El compartimento B tiene 50 cm de longitud y contiene inicialmente agua a 10 bar y 700 ºC. La sección del cilindro y émbolo es de 0.1 m^{2}. Se retiran los topes y el émbolo pierde sus propiedades aislante comenzando a moverse lentamente hasta que la presión y la temperatura de ambos compartimentos se igualan. Calcular la presión y temperatura final y la distancia recorrida por el émbolo.


\section{Examen Julio 2016}
\subsection{TEORÍA}
\textbf{Pregunta 1}. (1 punto) Demuestre que el rendimiento de una Máquina de Carnot es sólo función de las temperaturas de los focos.
\\
\textbf{Pregunta 2}. (1 punto) Por una tobera circula Monóxido de Carbono (\gamma)$=1.3$ entrando en la misma a 650 kPa, 300 K y velocidad despreciable. Se pide:
\begin{enumerate}
    \item Presión, temperatura, densidad y velocidad de la sección crítica.
    \item Si la presión de la cámara donde desemboca la tobera es de 450 kPa ¿se alcanzarán las condiciones críticas? ¿cuál será el Mach a la salida?
\end{enumerate}
Nota: Considere el Monóxido de Carbono un gas ideal.
\\
\textbf{Pregunta 3}. (1 punto) En un recipiente adiabático se mezclan 100 g de hielo y 200 g de agua líquida a 0 ºC, y se introducen 10 g de vapor de agua a 1 atm de presión y 100 ºC. Determine la temperatura final y la variación de entropía generada resultante de la mezca (kJ/K).
\\
Datos: Calor específico del hielo: $c_{h} = 2.09 kJ/kg K$, calor específico del agua líquida $c_{L}= 4.2 kJ/kg K$, calor de fusión del hielo
$c_{F}=334 kJ/kg$, calor latente de vaporización $c_{V} = 2257 kJ/kg$.
\\
\textbf{Pregunta 4}. (1 punto) Una determinada masa de Cloruro de Metilo $(CH_{3}Cl)$ se encuentra inicialmente a 500 K y 60 bar. Se calienta a presión constante hasta que alcanza los 600 K. Determine el porcentaje de error que se comete al calcular la variación de volumen específico de la misma considerando gas ideal en lugar de emplear los factores de compresibilidad.
\\
\subsection{PROBLEMAS}
\textbf{Problema 1}. Un cilindro cerrado por un émbolo (que se desplaza sin fricción) tieen encima una masa que hace que la presión dentro del mismo permanezca constante e igual a 10 MPa. El cilindro contiene inicialmente
agua a 700 ºC ocupando un volumen de 100 litros. A continuación se enfría el agua hasta que pasa a ser líquido saturado y el calor extraído de ese proceso se suministra a una máquina térmica cíclica que interactúa con el medio ambiente (que se encuentra a 30 ºC). Si todo el proceso se considera resversible, determine el rendimiento de la máquina cíclica.
\\
\textbf{Problema 2}. Un motor Otto de 4 cilíndros y 4 tiempos tiene una cilindrada total de 1800 $cm^{3}$. La velocidad media del émbolo es de 10 m/s y el motor proporciona una potencia efectiva de 80 kW y un par de 142 Nm, siendo sus rendimientos volumétrico y efectivo 0.5 y 0.4 respectivamente. Considerando una densidad del aire de 1.289 kg/m^{3} y un poder calorífico inferior de 45000 kJ/kg, determine: el diámetro del émbolo, la presión media efectiva y el dosado.
\\
\textbf{Problema 3}. Un condensador está compuesto por una carcassa y una serie de tubos de bronce comercial en su interior. El diámetro interior de los tubos es de 1.35 cm y el exterior de 1.54 cm. El vapor de agua se condensa alrededor de los tubos y por el interior de los mismos circula agua líquida a una velocidad de 1.4 m/s entrando al condensador a una temperatura de 17 ºC y abandonándolo a 37 ºC. Si la tasa de suministro de agua de enfriameinto es de 55000 kg/h. Determine: 1. el coeficiente global de transferencia de calor del lado interior de los tubos (W/m^{2}K), 2.
el número de tubos del condensador.\\
Nota: Desprecia la resistencia por convección del lado del vapor que condensa.  


\end{document}