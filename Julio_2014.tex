\documentclass[12pt]{article}
\usepackage[utf8]{inputenc}
\usepackage[spanish]{babel}

%Fuente (compilarlo en Latex.pdf normal porque va más rápido y luego
%y al final insertarlo en Arial) DA ERROR
%\usepackage{fontspec}
%\setmainfont{Arial}

%Estructura de la Página
\usepackage[left=2.54cm,right=2.54cm,top=2.54cm,bottom=2.54cm]{geometry}

%Páginas en horizontal


%Pies de página y encabezado
\usepackage{fancyhdr}

%Comentario párrafos
\usepackage{verbatim}

%Paquete arquitectura página
\pagestyle{fancy}
\fancyhf{}
\lhead{José Honrubia Blanco}
\rhead{Exámenes Ingeniería Térmica.}
\rfoot{\thepage}
\lfoot{Academia David Martínez}

%Interlineado
\renewcommand{\baselinestretch}{0.5}



%Paquete Matemático
\usepackage{amsmath}
\usepackage{amsfonts}
\usepackage{amssymb}
\usepackage{breqn}

%Paquete para el código
% Paquetes Listing

\usepackage{listings}
\usepackage{xcolor}

%Settings 

\definecolor{codegreen}{rgb}{0,0.6,0}
\definecolor{codegray}{rgb}{0.5,0.5,0.5}
\definecolor{codepurple}{rgb}{0.58,0,0.82}
\definecolor{backcolour}{rgb}{0.95,0.95,0.92}


\lstdefinestyle{mystyle}{
 backgroundcolor=\color{backcolour},   
 commentstyle=\color{codegreen},
 keywordstyle=\color{magenta},
 numberstyle=\tiny\color{codegray},
 stringstyle=\color{codepurple},
 basicstyle=\ttfamily\footnotesize,
 breakatwhitespace=false,         
 breaklines=true,                 
 captionpos=b,                    
 keepspaces=true,                 
 numbers=left,                    
 numbersep=5pt,                  
 showspaces=false,                
 showstringspaces=false,
 showtabs=false,                  
 tabsize=2
}

\lstset{style=mystyle}
%Paquete Imágenes

\usepackage{graphicx}
\graphicspath{ {img/} }
\usepackage{subfig}

%Bibliografía
\usepackage[backend=bibtex]{biblatex}
\addbibresource{referencias.bib}

%Espacio entre párrafos
\setlength{\parskip}{0.5cm}
%Sangría
\setlength{\parindent}{0cm}

\title{Exámenes Ingeniería Térmica}
\author{José Honrubia Blanco }
\date{Noviembre 2022}

\begin{document}

\section{Examen Julio 2014}
\subsection{TEORÍA (NO HACE FALTA USAR LAS TABLAS)}
\textbf{Pregunta 1}. (1 punto) Un dispositivo pistón-cilindro contiene 50kg de agua a 250 kPa y 25 ºC. El área de la seeción transversal del pistón es 0,1 m^{2}. Transferimos calor al agua, haciendo que parte de ella se evapore. Cuando el volumen alcanza los 0.2 m^{3}, el pistón choca con un muelle lineal cuya constante elástica es 100 kN/m. Seguimos transfiriendo calro al agua hasta que el pistón se eleva 20 cm más. Determine: 1. Representar el proceso en un diagrama P-V, 2. presión y temperatura final y 3. trabajo realizado durante este proceso. 
\\ 
\textbf{Pregunta 2}. (1 punto) Un motor de un buque funciona según un ciclod e Carnot que extrae calor del agua del mar a 18 ºC y cede una parte a un depósito de hielo seco a $-$78 ºC. Sie l motor debe desarrollar 8000 CV de potencia, ¿cuánto hielo seco se consumirá durante la marcha de un día? El calor de sublimación del hielo seco es de 137 cal/g.
\\
\textbf{Pregunta 3}. (1 punto) sor toma 1 kg/s de aire a 1 bar a 25 ºC comprimiéndolo hasta 8 bar y 160 ºC. La transferencia de calor a su entorno es de 100 kW. Calcular la potencia consumida (kW) y defínase y evalúese la eficiencia exergética.
\\
\textbf{Pregunta 4}. (1 punto) Se introduce helio en una tobera adiabática que oepra en estado estacionario. Las condiciones a la entrada de la tobera son: 1300 K, 4 bar y velocidad de 10 m/s. A la salida, la temperatura del helio es de 900 K, la presión de 1.45 bar. Determinar: 
1. la velocidad a la salida (m/s), 2. rendimiento de la tobera, 3. exergía destruida al pasar el gas a través de la tobera (kJ/kg). \\
\textbf{Hipótesis:} Considerar el helio gas perfecto, $T_{0} = 20ºC$ y $P_{0} = 1 bar$.

\subsection{PROBLEMAS}
\textbf{Problema 1}. (\textbf{2 puntos}) Un cilindro rígido y adiabático, dispuesto horizontalmente, está dividido en dos compartimentos por un émbolo muy delgado y aislante, sujeto inicialmente por unos topes. El compartimento A (el de la izquierda)tiene 10 cm de longitud
y contiene iniucialmente agua a 20 bar y 250 ºC. El compartimento B tiene 50 cm de longitud y contiene inicialmente agua a 10 bar y 700 ºC. La sección del cilindro y émbolo es de 0.1 m^{2}. Se retiran los topes y el émbolo pierde sus propiedades aislante comenzando a moverse lentamente hasta que la presión y la temperatura de ambos compartimentos se igualan. Calcular la presión y temperatura final y la distancia recorrida por el émbolo.

\end{document}